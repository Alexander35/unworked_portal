\section{UTF-8}

\subsection{Erlang}

The main thing you should know about Erlang unicode is that

\vspace{1\baselineskip}
\begin{lstlisting}
unicode:characters_to_binary("UniText") == <<"UniText"/utf8>>.
\end{lstlisting}
\vspace{1\baselineskip}

I.e. in N2O DSL you should use:

\vspace{1\baselineskip}
\begin{lstlisting}
#button{body= <<"Unicode Name"/utf8>>}
\end{lstlisting}
\vspace{1\baselineskip}

\subsection{JavaScript}

Whenever you want to send to server the value from DOM element
you should use utf8\_toByteArray.

\vspace{1\baselineskip}
\begin{lstlisting}
> utf_toByteArray(document.getElementById('phone').value);
\end{lstlisting}
\vspace{1\baselineskip}

However we created shortcut for that purposes which knows
about radio, fieldset and other types of DOM nodes. So you should use just:

\vspace{1\baselineskip}
\begin{lstlisting}
> querySource('phone');
\end{lstlisting}
\vspace{1\baselineskip}

querySource JavaScript function ships in nitrogen.js which is part
of N2O JavaScript library.

\paragraph{}
Whenever you get unicode data from server you should prepare it before place
in DOM with utf8\_decode:

\vspace{1\baselineskip}
\begin{lstlisting}
> console.log(utf8_decode(receivedMessage));
\end{lstlisting}
\vspace{1\baselineskip}
